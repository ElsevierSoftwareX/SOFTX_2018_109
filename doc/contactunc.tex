%--------------- uncertainties due to contact point
 
 \subsection{Uncertainty due to choice of contact point}
 It may not be always clear where exactly the contact between the indenter and the sample occurs. This induces an uncertainty of type B which must be estimated.
 In order to facilitate this, we explicitly show how the results change when the contact point was chosen to be more to the left or to the right by a certain amount of points. 
 Zero contact point shift means the original contact point was used. A positive (negative) contact point $N$ shift means the $N$th neighboring point to the right (left) was used instead. 
 This corresponds to a shift by $(\Delta h, \Delta F)$ in the unloading data; data are added or removed from the loading data as well as shifted by $(\Delta h, \Delta F)$.
 The ranges for the interval of the fitting procedures are transformed. If they were chosen in the length regime, either by mouse or by input in the entries, they are shifted by $\delta h$.
 Percentages of the maximum forces are not transformed, since the maximum force is already shifted. 
 
